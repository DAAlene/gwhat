\documentclass[TechnicalNoteMeteo.tex]{subfiles}

\begin{document}

\subsection{Results and Discussion}

Tests have shown that inclusion of more than four stations does not significantly improve the interpolation and may in fact degrade the estimate. 

The quality of the estimates is strongly affected by seasonality. Stations at higher elevations are difficult to estimate accurately, in large part because of the topographical diversity of the surrounding stations leading to degradation of spatial coherence among stations.

The tendency for all of the methods to have a negative bias is indicative of the nature of precipitation distributions to be positively skewed (interpolated values will tend to cluster about the median error rather than the mean).

According to \cite{xia_forest_1999}, the two most important factors in climatology are the inter-correlations in the station network, and the seasonal variations in the relations between the stations.

\end{document}

%The Canadian Daily Climate Database (CDCD), owned by the Government of Canada, contains daily data for air temperature and precipitation dating back to 1840 to the present for about 8450 stations distributed across Canada. Data can be downloaded manually on the Government of Canada website (\url{www.climate.weather.gc.ca}) for each year individually and saved in a csv file. This process involves a lot of repetitive manipulations and is a time consuming task. Moreover, the re-organization of the individual data files, saved for each year separately, in a more convenient format can also represent a tedious task when done manually. Alternatively, it is possible to order a DVD containing the entire database for a small fee. This option has the disadvantage of only providing an image in time as data cannot be updated.  

%This region has been the subject of an extensive characterization project within the `Programme d'acquisition de connaissances sur les eaux souterraines du Québec' (PACES) whose main objective was to prepare a realistic and concrete picture of the groundwater resources for the region \citep{carrier_portrait_2013}.