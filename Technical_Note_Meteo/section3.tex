\documentclass[TechnicalNoteMeteo.tex]{subfiles}

\begin{document}

The code is compliant with either Python 3.4 or 2.7.9 or later. It requires numpy , xlrd and PySide or any later version of these librairies. The algorithm is organized as a base class of the Qt GUI Framework using the PySide binding. Signals are also emitted at various stade in the gap-filling routine. This has been done to facilitate the addition of a Graphical User Interface on top of the algorithm with the Qt GUI Development framework. There is an mininimal working example of application that is documented at the end of the file with the algorithm at the end of the file.

The algorithm is also implemented in the free and open source software WHAT which provides a user friendly and convenient interface. Detailed information about the use of the algorithm with the interface of WHAT are provided in the user guide of WHAT.

\subsection{Parameters}\label{sec:parameters}

\subsection{Input Data}

It is possible to use weather data from any sources in WHAT, given the right format is used, either to fill the gaps in the weather time series and/or to interpret water level time series. For this purpose, it is recommended to use a copy of one of the sample files that are provided in the project example (distributed with the software) and fill the information and the data directly in it. The file must be kept in a text format using tab-separated values either with the extension ‘‘.csv’’ or ‘‘.out’’, depending if you want to fill the gaps in the weather time series or  interpret water level time series. This can be achieved with any standard spreadsheet application such as Microsoft Excel or LibreOffice Calc. The format of the header must be faithfully observed for those files. In addition, ``NaN'' values must be entered where data are missing. Data must also be in chronological order, but do not need to be continuous over time. That is, missing blocks of data (e.g., several days, months or years) can be completely omitted in the time-series. These missing blocks of data will be filled during the gap filling procedure or will be ignored for the plotting of the hydrograph.

\subsection{Output}\label{sec:output}

\end{document}

%It the class for the algorithm comes with two companion class:. The first class is used to load the weather data into memory, align correctly the data series in regard to time, and do some quality check on the data. The class is used to hold all relative information regarding the target station, including the correlation coefficient calculated between the target station and its neighbors.
