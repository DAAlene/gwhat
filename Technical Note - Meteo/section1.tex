\documentclass[ARTICLETHERMIC.tex]{subfiles}

\begin{document}

Climate data are useful in several fields of Earth sciences, including hydrology, hydrogeology and agronomy. For this purpose, the Canadian Daily Climate Database (CDCD), owned by the Government of Canada, contains daily data for air temperature and precipitation dating back to 1840 to the present for about 8450 stations distributed across Canada. Data can be downloaded manually on the Government of Canada website (\url{www.climate.weather.gc.ca}) for each year individually and saved in a csv file. This process involves a lot of repetitive manipulations and is a time consuming task. Moreover, the re-organization of the individual data files, saved for each year separately, in a more convenient format can also represent a tedious task when done manually. Alternatively, it is possible to order a DVD containing the entire database for a small fee. This option has the disadvantage of only providing an image in time as data cannot be updated.  

Furthermore, climate datasets are, most of the time, incomplete. This can represent a major hindrance in various applications, such as for the use of hydrological or hydrogeological models that heavily depend on these data. Filling the gaps in weather datasets can quickly become a tedious task as the size of the data records and the number of stations increase. Moreover, it can also be quite complex when aspects such as time-efficiency of the method and accuracy of the estimated missing values are taken into account. This is particularly true for the estimation of missing daily precipitation data because of their high spatial and temporal variability \citep{simolo_improving_2010}. Although there exist various methods to estimate missing daily weather data that are well covered in textbooks and technical papers, few tools to perform this task efficiently and conveniently are available. 

WHAT (Well Hydrograph Analysis Toolbox), is a computer program that addresses the aforementioned issues \citep{gosselin_user_2015}. Firstly, it provides a graphical interface to the online CDCD that allows to search for stations interactively using location coordinates, download the available data for the selected weather stations, and automatically organize the data in a more convenient format. Secondly, the program also includes an automated, robust, and efficient method to quickly and easily fill the gaps in the daily weather datasets downloaded from the CDCD. In addition to the handling of missing data, WHAT includes a cross-validation resampling technique to conveniently validate and assess the uncertainty of the estimated missing values.


%This paper presents a computer program, 
This paper presents the algorithm that was developed as part of the WHAT software. The operation of the user interface of the software is provided in \cite{gosselin_what_2015}, available for download at this address: \url{https://github.com/jnsebgosselin/WHAT}.

%In addition to the handling of missing data, WHAT includes cross-validation resampling technique, to conveniently validate and assess the uncertainty of the estimated missing values. This feature represents a valuable tool for quickly assess the performance of the method for a given study area.

\end{document}