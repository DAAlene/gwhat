\documentclass[WHATMANUAL.tex]{subfiles}

\begin{document}

WHAT is free software: you can redistribute it and/or modify it under the terms of the GNU General Public License as published by the Free Software Foundation, either version 3 of the License, or (at your option) any later version.

This program is distributed in the hope that it will be useful, but WITHOUT ANY WARRANTY; without even the implied warranty of MERCHANTABILITY or FITNESS FOR A PARTICULAR PURPOSE. See the GNU General Public License for more details.

You should have received a copy of the GNU General Public License along with this program. If not, see \url{www.gnu.org/licenses}.

\vspace{1cm}

%\noindent
%WHAT is entirely written in the Python 2.7 programming language using the scientific Python development environment Spyder in Ubuntu Linux:
%
%\begin{description}
%\item \textbf{Python 2.7}. Copyright \textcopyright\space 2001-2015 Python Software Foundation.
%\item \textbf{Spyder}. Copyright \textcopyright\space 2012-2013 Matplotlib Development Team.
%\item \textbf{Ubuntu}. Copyright \textcopyright\space 2015 Canonical Ltd. 
%\end{description}
%
%\vspace{0.5cm}
%
%\noindent
%WHAT is based on the use of following third party packages:
%\begin{description}
%\item \textbf{PSwarm}. Copyright \textcopyright\space 2009 A. Ismael F. Vaz and L. N. Vicente.
%\item \textbf{Matplotlib}. Copyright \textcopyright\space 2012-2013 Matplotlib Development Team.
%\item \textbf{Numpy}. Copyright \textcopyright\space 2012-2013 Matplotlib Development Team.
%\end{description}
%
%\noindent
%This document has been written using LaTeX in the  integrated development environment (IDE) TextStudio. All images other were produces using either GIMP and Inkscape.
%\begin{description}
%\item \textbf{LateX3} Copyright \textcopyright\space LaTeX3 Project Team
%\item \textbf{TeXstudio} Copyright \textcopyright\space Benito van der Zander, Jan Sundermeyer, Daniel Braun, Tim Hoffmann.
%\end{description}
%
%\chapter*{Acknowledgements}
%
%WHAT has been funded in part by:
%
%\begin{description}
%\item CRSNG through a PhD grant to Jean-S\'ebastien Gosselin
%\item Projet Montérégie Est PACES
%\item CRSNG fund
%\item CGC
%\end{description}
%
%We would like to thank all people who have used WHAT since its earliest stages and provided essential technical feedback, constructive criticism, and helpful comments or have helped in the shaping of the science that lies under the hood of the software. In particular, special thanks to (in alphabetical order):
%
%\begin{description}
%\item Erwan Gloaguen, Professor of Geophysics and Geostatistics, INRS-ETE, Quebec, QC, Canada.
%\item Harold Vigneault, Research Professional, INRS-ETE, Quebec, QC, Canada.
%\item Marc Laurencelle, PhD Student in Earth Sciences, INRS-ETE, Quebec, QC, Canada.
%\item Pierre Ladev\`eze, PhD Student in Earth Sciences, INRS-ETE, Quebec, QC, Canada.
%\item Ren\'e Lefebvre, Professor in Hydrogeology, INRS-ETE, Quebec, QC, Canada.
%\item Xavier Mallet, Research Technician, Geological Survey of Canada – Quebec Division, QC, Canada.
%\end{description}
%
%\chapter*{Foreword}
%
%
%WHAT started originally as a little piece of code to fill missing data in .
%
%This document is divided in two parts: User Manuel and Theoretical Basis.
%
%The part User Manual contains all the information relative to the use of the software.
%
%Theoretical Basis contains literature review, mathematics and theoretical development and theoretical concepts on which the tools included in WHAT are based on.
%
%Technical Documendation contains everything that is programming related. The entire code is presented in flowchart and the structure and organization of the source files.
\end{document}