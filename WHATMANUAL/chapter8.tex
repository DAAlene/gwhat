\documentclass[WHATMANUAL.tex]{subfiles}

\begin{document}

\chapter{Estimation of missing daily climatological data}\label{chap:Missing_weather_theory}

\section{Introduction}

Climate data are useful in several fields of Earth Sciences, including hydrology, hydrogeology and agronomy. However, climate datasets are often incomplete. This represents a serious issue to the use of models that depend on these data. For example, the method used for the estimation of groundwater recharge in WHAT requires serially complete (no missing data values) air temperature and precipitation daily time series. The quality of the groundwater recharge assessments made with this technique are highly dependent upon the quality of the weather date set used. Hence, the adequate and accurate handling of missing data has been made an integral part of groundwater recharge assessment in WHAT.

A guide for the preparation of gapless weather data series has already been presented in Section~\ref{chap:gapfilling}. This section presents a brief literature review on the subject, followed by a detailed description of the method implemented in WHAT for the estimation of missing daily weather data, and finally an application is presented for the Monteregie Est region, located in southern Quebec, Canada. 

\section{Literature review}

Filling the gaps in weather data time series is a tedious task that can become rather complex when efficiency of the method, accuracy in the estimation of the missing values, and preservation of the statistical properties of the weather time series (probability distribution and normals) are required. This is particularly true for the estimation of missing daily precipitation data because of their high spatial and temporal variability \citep{simolo_improving_2010}. 

The creation of a serially complete weather dataset generally consists in the replacement of missing daily data with estimated values calculated from simultaneous observations at nearby stations. Within-station method, i.e. methods that only use data from the series being filled, have been proven to perform badly compared to methods based on the use of data from neighboring stations and are moreover ill-adapted for the reconstruction of daily precipitation time series \citep{eischeid_quality_1995,simolo_improving_2010,kemp_estimating_1983}.

Numerous spatial interpolation techniques exist for handling the missing data in the weather time series of a given station by using data from irregularly spaced neighboring station (e.g. simple arithmetic averaging, inverse distance method, single best estimator and multiple regression analysis). \cite{eischeid_creating_2000} demonstrated that the multi-linear regression (MLR) method \citep{degaetano_method_1995} outperformed most of the commonly used techniques among the six classical methods tested for the creation of a serially complete daily temperature and precipitation dataset for the United States. Similarly, \cite{xia_forest_1999} also found that the MLR method was consistently the most accurate out of the six classical methods tested to estimate missing daily weather data in Bavaria, Germany.

However, weighing and regression-based techniques, including the MLR method mentioned above, all tend to overestimate the number of rainy days. Also, the rainfall probability distribution is usually not preserved with these techniques, that is heavy precipitation events are systematically underestimated. \cite{simolo_improving_2010} have proposed a two-step procedure to modifies the MLR method to address these issues.

In recent years, the estimation of missing daily weather data using artificial intelligence techniques have been the subject of many research studies \cite{kashani_evaluation_2011,kim_spatial_2008,srikanthan_comparison_2005,coulibaly_comparison_2007,abebe_application_2000,teegavarapu_improved_2005}. \cite{kashani_evaluation_2011} found that these new data-driven methods, more specifically the artificial neural network and the genetic programming techniques, outperformed the classical application of the MLR approach for three different climates in Iran. Artificial intelligence techniques are interesting because, unlike the classical weighing and regression-based methods, they are able to reflect the inherently stochastic nature of natural processes. Nevertheless, the results of \cite{kashani_evaluation_2011} also indicated that the MLR method was an appropriate technique, among the eight traditional methods tested, for all the weather variables (air temperature, wind speed, relative humidity, and precipitation) and the different climate conditions (dry to extra humid conditions) tested \citep{kashani_evaluation_2011}.

\section{Description of the method}

Though there exist various methods to estimate missing daily weather data that are well covered in textbooks and technical papers, tools to perform this task quickly, efficiently and easily are scarce. The implementation of some of the more complex (and often most accurate) methods discussed in the section above can be a daunting task for a research study that is not directly related to the fields of meteorology or to the geostatistics. Because of time or resources constraints, it can be tempting in these studies to handle the problem of missing daily weather data with a simple method that is not always appropriate to the objective of the study. This is the case for example of using a simple and practical within-station approach to quickly and easily estimate missing values for daily air temperature and precipitation which is not an appropriate method for these kind of data \citep{eischeid_quality_1995,simolo_improving_2010,kemp_estimating_1983}. For this reason, the creation of gapless weather data series (air temperature and precipitation) has been made an integral part of the software WHAT.

The method for filling the gaps in the weather time series of air temperature and precipitation is based on the implementation of the classical MLR method presented in \cite{eischeid_creating_2000}. The rationales behind this choice was to provide in WHAT a robust, efficient, accurate, and well known method that was adequate to most conditions (e.g. climate, land cover, topography). Based on the literature review presented in the section above, this choice appears to be a very good default option for the first a version of the software. Since the code for loading and manipulating the entire weather dataset of a given study area is object oriented, it should be relatively easy to add modifications to the currently available approach such as the additional steps for the MLR method proposed in \cite{simolo_improving_2010} or to completely implement one of the new artificial intelligence techniques that were developed in recent years \citep{kashani_evaluation_2011}.

In addition to the handling of missing data, WHAT includes the framework to easily validate and assess the uncertainty of the estimated missing values with a jackknife resampling technique. This feature is an invaluable tool for quickly validating the method for a given study area and to test the performance of new methods that could be added to future version of the software. The entire gapfilling process used in WHAT is illustrated in the flowchart of Figure~X.

\subsection{Step 1: pre-selection of the stations}

The quantity and the selection of surrounding stations are critically important for accurate estimation of missing weather data \citep{eischeid_quality_1995}.


Problems arise though because of synchronize missing values in the target and neighboring weather station data series that varies trough time. The process to determine which stations are to be used for the estimation of missing data must then be done for each day for which data are missing.

In order to determine which stations are to be used, surrounding stations are preselected based on their relationship with the target station.

The 15 closest stations are identified for each target station and are ranked by the value of the correlation coefficient between the candidate station and its neighbors.

The number (never greater than four) of neighboring stations meeting the criteria is not fixed in time. It varies depending on available station data for the year, month, day in question. As such, the interpolation models may also change in time.

The quality of the estimates is strongly affected by seasonality. Stations at higher elevations are difficult to estimate accurately, in large part bacause of the topographical diversity of the surrounding stations leading to degradation of spatial coherence among stations.

According to an extensive literature review and the experience of Tronci et al (1986), the influence radius was chosen as 100 km in our study. XIA

\subsection{Step 2: Estimating Missing Daily Value}

This project \citep{eischeid_creating_2000} has the objective to create serially complete daily datasets in a systematic, well-documented fashion that can be utilized for many hydrologic and other natural resource conservation models. The objective of this project is to create a serially complete (no missing data values) daily temperature and precipitation dataset

Quality control performed by NCDC on this dataset included a procedure (Reek et al. 1992).

The method validation was performed with reference to the rain-gauge network presented in Section 1, by using a jacknife-like procedure, that involves the removal of subsets of data from the target series before reconstruction is carried out. This avoids ``self-influence'' of the observations that are being estimated. Specifically, one year of the target series ar a time was fully discarded, together with a n-year long window centered on that year, for fixed n, and subsequently reconstructed. Imputed data were finally compared with the original (removed) ones to assess the accuracy of the results.

The weather network used to test the method implemented in WHAT is located in the Monteregie Est regions located in the province of Quebec, Canada. This region feature strongly variable topography and land use. The netwrok is presented in figure X.

Also, stations from bordering states were extracted to improve the spatial distribution of sites surrounding target stations located near state borders.

The 22 states reflect a wide variety of terrain and a diversity of climatic regimes, which allows a means for testing the efficacy of daily estimates for regional and seasonal differences. In addition, with few exceptions, the geographic distribution across the western states is relatively uniform, which provides a stable estimation environment and a substantial number of serially complete stations for natural resource modeling.

This is basically a two step method. The first step consist in selecting the data series with the best corellation coefficient. The second steo consiste in building a MLR model and estimate the missing values.

The tendency for all of the methods to have a negative bias is indicateive of the nature of precipitation distributions to be positively skewed (interpolated values will tend to cluster about the median error rather than the mean) \citep{eischeid_quality_1995}.

In order to determine which stations are to be used, surrounding stations are preselected based on their relationship with the target station. The 15 closest stations are identified for each target station and are ranked by the value of the correlation coefficient between the candidate station and its neighbors. A minimum of one station is needed to compute the estimate at the target station, with a maximum of four. Tests have shown that inclusion of more than four stations does not significantly improve the interpolation and may in fact degrade the estimate.

The number (never greater than four) of neighboring stations meeting the criteria is not fixed in time. It varies depending on available station data for the year/month/day in question. As such, the interpolation models may also change in time.

\subsubsection{Ordinary Least Square Criteria}

\subsubsection{Least Absolute Deviations Criteria}
REG can account for the local effect (i.e. topography, forest) particularly for forest climate stations.


The method of multiple regression using the least absolute deviations criteria (MLAD) is a robust version
of the general linear least squares estimation. The method of least squares is an effective method when the errors are normally distributed and independent. However, for precipitation data especially, the assumption of normality over the wide range of situations can lead to poor estimations. The principal advantage of least absolute deviations is its resistance to outliers and to overemphasis of large-tailed distributions (Barrodale and Roberts 1973). MLAD estimates the unknown parameters in a stochastic model so as to minimize the sum of absolute deviations of the neighboring station observations from the values predicted by the model. Regression coefficients b are calculated so as to minimize a set of n measurements on m surrounding stations (independent variables), anote the associated measurement on the dependent (target station) value. The linear programming techniques of Barrodale and Roberts (1973) are used to accomplish this task.

WHAT have laid the basis and foundation of a method that allows easy comparison of the method and allows to immediately determine if a change in the methodology has resulted in an improvment in the estimation errors.

The goal of any quality control procedure is to provide the end user with as much information as possible, such as he/she can make an informed choice whether to accept or reject a particulat monthly value. The results of our analysis provide the end user with a set of flags, noted above, an with summary statistics of the efficacy of several estimation procedures.

\subsection{Step 3: Quality Control of the Data}

Prior to the analysis of climatic time series, it is important to remove data outliers in a methodical manner.

\cite{eischeid_quality_1995} presented an objective quality control analysis scheme that examined global climate data for outliers temporally and spatially.

A temporal check and a spatial check for outliers are described in the next section. When combined, these two measures of quatlity control provide a comprehensive set of uncertainty flags to determine the validity of a particular monthly value.

After estimating daily maximum and minimum temperatures, a series of internal consistency checks were
performed to ensure that estimates did not violate obvious constraints associated with recording maximum
and minimum temperatures. Typical tests include identifying estimated maximum temperatures lower than a
previous day’s minimum and an estimated maximum lower than a minimum for the same day. These inconsistencies were corrected by assigning corrected maximums or minimums where appropriate
or averaging the maximum and minimum temperatures for the previous and subsequent days.

\subsection{Step 4: Validation}

According to \cite{xia_forest_1999}
In climatology, the two most important factors are the inter-correlations in the station network, and the seasonal variations in the relations between the stations. {XIA}
In order to test the efficacity of the estimation techniques, each of the six interpolation methods is compared with respective nonmissing observations.

The accuracy of the daily precipitation estimate is dependent on the quality and quantity of the surrounding stations utilized to estimate a value at a particular site. The determination of daily precipitation totals is much more sensitive to these factors than air temperature, particularly with regard to the elevation of the site to be estimated.

\section{Monteregie Est study case}

\subsection{Materials and Method}

Tests have shown that inclusion of more than four stations does not significantly improve the interpolation and may in fact degrade the estimate. 

\subsection{Results and Discussion}

\section{Future Work}

\end{document}