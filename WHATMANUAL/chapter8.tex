\documentclass[WHATMANUAL.tex]{subfiles}

\begin{document}

\chapter{Estimation of missing daily climatological data}

Estimation of missing climatological data is an important task. It is particularly important in mountain and forest regions where meteorological stations are scarce, and the observed climatological data are strongly influenced by topography and the forest microclimate.

Climate data are useful in several areas, including hydrology, hydrogeology and agronomy. However, climate data sets are often incomplete. The estimation of missing data can be a complex and tedious task. This is particularly the case for daily precipitation because of their high spatial and temporal variability. A user friendly, menu-driven, and interactive computer program for rapid and automatic completion of daily climatological series has been developed. Missing data for a given weather station are estimated using a multiple linear regression model, generated using data from nearby stations. For daily precipitation, it is possible to activate an option that forces the algorithm to preserve the probability distribution of data. This is an advantage over conventional approaches that tend to overestimate the number of wet days and underestimate the high intensity precipitation events. The software also allows downloading and automatic formatting of raw data available on the Environment Canada website. The software is demonstrated for two weather station located in Monteregie Est region, southern Quebec. Cross-validation was used to check the method and to define the optimal parameters to minimize the error in estimating missing daily precipitation.

\end{document}