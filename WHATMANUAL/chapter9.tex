\documentclass[WHATMANUAL.tex]{subfiles}

\begin{document}

\chapter{MRC Estimation}

The Master Recession Curve (MRC) is a mathematical expression describing the average behavior of the declining water-table at a particular site during periods when groundwater recharge is negligible. The strength of this correlation and thus the validity of the MRC depends on many factors, including the local topography, seasonality of the climate, hydraulic properties of the subsurface porous medium, and depth to the water-table \citep{heppner_computer_2005}. 

Estimation of the MRC from a well hydrograph is done in WHAT in two step. The first step consists in the determination of the local extremum in the hydrograph in order to identify the period of time when the water-table recedes and the groundwater recharge is supposed negligible. 

La seconde étape consiste à l'optimization des paramètres de l'équation mathématique décrivant la MRC. The MRC takes the form of a linear relation between water-table elevation and water-table declining rate. The resulting water-table recession hydrograph has the familiar exponential decline shape (MONTRER UN EXEMPLE). :

Since the MRC can varie with seasonality of the climate, a different MRC can be estimated for the winter and summer. This option is not yet implemented in the code but will be shortly.

Water-table fluctuation principle states that recharge is the product of water-table rise and the specific yield. variation of water-table due to recharge can be difficult to estimate because water-tables typically are in transient state of decline between storm or recharge events. predict what the water-table elevation would be at the succeeding time step in the absence of recharge. The predicted water-table elevation is compared to the measured water-table elevation and the difference between these 2 values is multiplied by the specific yield to get recharge for that time step. The user is responsible for providing the best coefficient for the MRC function.

\section{Local Extremum Search}

The position of the local extremum in the water level time series can be estimated with the use of an automated algorithm that is based on a method presented in Chapter 6 of \cite{vamos_automatic_2012}.

% [Décrire brièvement en une phrase ou deux l'essence de l'approche]

Among the resulting receding segments, only the segments with a time span of more than a week are kept, and the shorter segment are disregarded. This is to remove receding segment that are due to noise.  In any case, it is best to inspect 
Dans un premier temps, l'hydrogramme de puits est filtré par une moyenne mobile centrée avec une période d'une journée afin de réduire le bruit haute-fréquence. En second lieu, tous les maximums et minimums locaux sont localisés par une approche automatisée dans Matlab. Une inspection visuelle est ensuite réalisée dans le but d'éliminer les faux extrémums ou encore pour identifier des extrémums qui auraient été omis par l'algorithme.

\section{Parameter Optimization}

\begin{equation}
   \frac{\partial h}{\partial t} = ah + b
\end{equation}

\begin{equation}
   h^{t+1} = \left[\left(1 + \frac{\delta a \Delta t}{2}\right)h^t + b \Delta t\right]\left[1 - \frac{\delta a \Delta t}{2}\right]^{-1}
\end{equation}

Cette approche a été utilisée pour l'estimation de la recharge notamment par Delin et al. (2007), Heppner et Nimmo ainsi que Crosbie et al. (2005). Plusieurs programmes sont disponibles pour la détermination automatique d'une CMR à partir d'un hydrogramme de puits, dont les algorithmes développées par Crosbie et al. (2005), Posavec (2006) et Heppner et Nimmo (2005). 
La Error: Reference source not found montre un exemple d'une courbe maîtresse de récession obtenue pour le puits P0-19 (Mont Rougemont) avec le programme Visual Basic de Posavec (2006) utilisant une méthode automatisée d'appariement par bande.

La CMR ainsi définie est par la suite utilisée pour tracer la projection de la courbe de récession antécédente au lieu de procéder manuellement. Une estimation de la recharge peut ensuite être calculée directement par l'application de l'équation [1.

Cette technique à l'avantage d'éliminer une bonne partie de la subjectivité de l'utilisateur. De plus, contrairement à l'approche graphique, elle permet facilement d'estimer la recharge en période de récession de la nappe. Dans la littérature, l'approche CMR est généralement utilisée pour calculer la recharge pour tous les points de mesure constituant l'hydrogramme de puits. Cela rend cette approche très sensible au bruit qui est souvent observable dans les mesures de niveaux d'eau. (EXEMPLE– À VENIR) Ce bruit peut être attribuable par exemple aux variations de la pression atmosphériques et aux marées terrestres (Rasmussen et Crawford, 1997; Butler et al., 2011), à du pompage, à l'effet Lisse (Weeks, 2002; Crosbie et al., 2005), aux variations de températures ou encore par l'appareil de mesure même.

La réorganisation des termes et l’intégration de l’équation
 par rapport au temps entre deux points (tA,hA) et (tB,hB) d'un hydrogramme de puits, tel que défini à la Error: Reference source not found, résulte en l’équation suivante:
 
\section{Groundwater Recharge Estimation}

\end{document}